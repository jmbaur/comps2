\documentclass[%
 aip,
%jmp,%
% bmf,%
% sd,%
rsi,%
 % amsmath,amssymb,
% preprint,%
 reprint,%
%author-year,%
%author-numerical,%
]{revtex4-1}

\usepackage[utf8]{inputenc}
% \usepackage{graphicx}% Include figure files
% \usepackage{dcolumn}% Align table columns on decimal point
% \usepackage{bm}% bold math

\begin{document}
\title{Comps Presentation Outline} %Title of paper
\author{Jared Baur}
\date{\today}

\maketitle

\section{Introduction}
Double pendulum is a well known example of chaotic behavior.\\
\indent First half cycle is relatively predictable and repeatable, but after that chaos is prevalent.\\
\indent Can a baseball bat (or any other swinging element) be designed so that all the energy from the first pendulum arm (arms of person swinging bat) can be transferred to the bat?

\section{Chaos\cite{Shinbrot1992}}

\begin{equation}
	\Delta x(t) \sim \Delta x(t_0) e^{\lambda t}
\end{equation}

Equation 1 describes the separation between nearby trajectories (so this describes the difference between expected values and actual values.\\
\indent As time increases, the chaos increases.\\
\indent Show tandem double pendulum to show unpredictability\\
\indent pendula that are released from a larger initial angle (or a bat that is swung after traveling a large distance from the arms) produce trajectories that quickly diverge from one another

\subsection{Cause of Exponential Growth}

\section{Equations of Motion}
Explain why Newtonian equations are used instead of Lagrangian equations\\
\indent Talk about moment of inertias for both pendulum arms\\
\indent Talk about initial conditions for different situations

\section{Experimental Results}
Talk about procedure taken to produce experimental results

\section{Energy Transfer}
Center of percussion for bats \cite{Cross2005}\\
\indent Building the perfect bat that transfers all energy

\section{Relevance to other applications}
Overarm throwing (upper arm is first pendulum arm and lower arm is second pendulum arm)

\bibliography{main}

\end{document}
