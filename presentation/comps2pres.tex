% Copyright 2004 by Till Tantau <tantau@users.sourceforge.net>.
%
% In principle, this file can be redistributed and/or modified under
% the terms of the GNU Public License, version 2.
%
% However, this file is supposed to be a template to be modified
% for your own needs. For this reason, if you use this file as a
% template and not specifically distribute it as part of a another
% package/program, I grant the extra permission to freely copy and
% modify this file as you see fit and even to delete this copyright
% notice. 

\documentclass[tikz]{beamer}

\setbeamertemplate{caption}[numbered]
\setbeamercovered{invisible}
\usepackage{mwe,tikz}
\usepackage[percent]{overpic}
\usepackage{animate}
\usepackage{bm}
\usepackage{amsmath}
% \usepackage{onimage}
% There are many different themes available for Beamer. A comprehensive
% list with examples is given here:
% http://deic.uab.es/~iblanes/beamer_gallery/index_by_theme.html
% You can uncomment the themes below if you would like to use a different
% one:
% \usetheme{AnnArbor}
% \usetheme{Antibes}
% \usetheme{Bergen}
% \usetheme{Berkeley}
% \usetheme{Berlin}
\usetheme{Boadilla}
% \usetheme{boxes}
% \usetheme{CambridgeUS}
% \usetheme{Copenhagen}
% \usetheme{Darmstadt}
% \usetheme{default}
% \usetheme{Frankfurt}
% \usetheme{Goettingen}
% \usetheme{Hannover}
% \usetheme{Ilmenau}
% \usetheme{JuanLesPins}
% \usetheme{Luebeck}
% \usetheme{Madrid}
% \usetheme{Malmoe}
% \usetheme{Marburg}
% \usetheme{Montpellier}
% \usetheme{PaloAlto}
% \usetheme{Pittsburgh}
% \usetheme{Rochester}
% \usetheme{Singapore}
% \usetheme{Szeged}
% \usetheme{Warsaw}
\usetikzlibrary{arrows,decorations.markings,decorations.pathmorphing, patterns,shapes}

\title{The Double Pendulum}

% A subtitle is optional and this may be deleted
\subtitle{Creating the Perfect Baseball Bat}

\author{Jared Baur}
% - Give the names in the same order as the appear in the paper.
% - Use the \inst{?} command only if the authors have different
%   affiliation.

\institute[Occidental College] % (optional, but mostly needed)
{
  Physics Department\\
  Occidental College
% - Use the \inst command only if there are several affiliations.
% - Keep it simple, no one is interested in your street address.
}
\date{February 25, 2019}
% - Either use conference name or its abbreviation.
% - Not really informative to the audience, more for people (including
%   yourself) who are reading the slides online

% \subject{Theoretical Computer Science}
% % This is only inserted into the PDF information catalog. Can be left
% % out. 

% % If you have a file called "university-logo-filename.xxx", where xxx
% % is a graphic format that can be processed by latex or pdflatex,
% % resp., then you can add a logo as follows:

% % \pgfdeclareimage[height=0.5cm]{university-logo}{university-logo-filename}
% % \logo{\pgfuseimage{university-logo}}

% Delete this, if you do not want the table of contents to pop up at
% the beginning of each subsection:

% \AtBeginSubsection[]
% {
%   \begin{frame}<beamer>{Outline}
%     \tableofcontents[currentsection,currentsubsection]
%   \end{frame}
% }

% Let's get started
\begin{document}

\begin{frame}
  \titlepage
\end{frame}

\begin{frame}{Scope}
  \tableofcontents
  % You might wish to add the option [pausesections]
\end{frame}

% Section and subsections will appear in the presentation overview
% and table of contents.

\section{Introduction}

\begin{frame}{The Double Pendulum}
	\only<1> {	
    		\begin{figure}
		    \centering
		    \animategraphics[loop,controls,width=\linewidth,scale=0.6]{10}{doublependulum_gif/doublependulum-}{0}{297}
		\end{figure}
	}
	\only<2> {
    		\begin{figure}
		    \centering
		    \animategraphics[loop,controls,width=\linewidth,scale=0.6]{10}{comparison_gif/comparison-}{0}{99}
		\end{figure}

		Chaos increases exponentially
		\begin{equation}
			\Delta x(t) \sim \Delta x(t_0) e^{\lambda t}
		\end{equation}
	}
	\only<3> {
		Chaos is not as prevalent in a baseball swing, due to the swing only occuring in the first half cycle of a double pendulum
    		\begin{figure}
		    \centering
		    \animategraphics[loop,controls,width=\linewidth,scale=0.6]{10}{baseballswing_gif/baseballswing-}{0}{23}
		\end{figure}
		Can we design a "perfect" baseball bat?
	}
	% \only<4> {
	% 	Can we design a "perfect" baseball bat?
	% }
\end{frame}

\section{Equations of Motion}
	\only<1> {

	}


\section{Experimental Results}

\section{Energy Transfer}

\section{Applications}

% \section*{Resources}
\end{document}
